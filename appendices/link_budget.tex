%
% link_budget.tex
%
% Copyright (C) 2021 by SpaceLab.
%
% GOLDS-UFSC Documentation
%
% This work is licensed under the Creative Commons Attribution-ShareAlike 4.0
% International License. To view a copy of this license,
% visit http://creativecommons.org/licenses/by-sa/4.0/.
%

%
% \brief Link budget calculation appendix.
%
% \author Gabriel Mariano Marcelino <gabriel.mm8@gmail.com>
%
% \institution Universidade Federal de Santa Catarina (UFSC)
%
% \version 0.1.0
%
% \date 2021/02/06
%

\chapter{Link Budget Calculation} \label{anx:link-budget}

This appendix shows the link budget calculation of all the satellite links (including the radio links of the payloads). The used method was taken from \cite{larson2005} (section 13.3).

\section{Distance to Satellite at Horizon}

The distance to satellite at horizon (the maximum theoretical distance between the satellite and a ground station) can be calculated using \autoref{eq:horizon-distance}.

\begin{equation} \label{eq:horizon-distance}
d = \sqrt{2\cdot R_{e}\cdot h + h^{2}}
\end{equation}

Where:

\begin{itemize}
    \item $R_{e}$ = Earth radius = 6378 km
    \item $h$ = Satellite altitude = 600 km
    \item $d$ = Distance to sattellite at horizon
\end{itemize}

So, the distance to satellite at horizon is:

\begin{equation} \label{eq:horizon-distance-result}
d = \sqrt{2\cdot 6378\cdot 600 + 600^{2}} = \mathbf{2830\ km}
\end{equation}

%\section{Power at Receiver}

%\begin{equation}
%P_{r} = P_{t} + G_{t} + G_{r} - L_{p}\ [dB]
%\end{equation}


\section{Free-Space Path Loss}

The free-space path loss ($FSPL$) can be calculated using \autoref{eq:fspl}.

\begin{equation} \label{eq:fspl}
FSPL = \left( \frac{4\pi d f}{c} \right)^{2}
\end{equation}

Where:

\begin{itemize}
    \item $d$ = Distance between the satellite and the ground station
    \item $f$ = Radio frequency
    \item $c$ = Speed of light
\end{itemize}

The FSPL value in decibels can be calculated with \autoref{eq:fsbl-db}.

\begin{equation} \label{eq:fsbl-db}
    \begin{split}
        FSPL^{dB} & = 20\log\left(\frac{4\pi}{c}\right) + 20\log\left(d\right) + 20\log\left(f\right) \\
                  & = 32,45 + 20\log\left(\frac{d}{1\ km}\right) + 20\log\left(\frac{f}{1\ MHz}\right) \\
    \end{split}
\end{equation}

The minimum distance between the satellite and a ground station is the satellite altitude, in this case: 600 km. The maximum distance is the distance at horizon, defined by \autoref{eq:horizon-distance-result}.

\subsection{Beacon}

Considering the frequency of the beacon as 145,97 MHz, the minimum and maximum FSBL is:

\begin{equation}
FSPL^{dB}_{max} = 32,45 + 20\log\left(\frac{2830}{1\ km}\right) + 20\log\left(\frac{145,97}{1\ MHz}\right) = \mathbf{144,8\ dB}
\end{equation}

\begin{equation}
FSPL^{dB}_{min} = 32,45 + 20\log\left(\frac{600}{1\ km}\right) + 20\log\left(\frac{145,97}{1\ MHz}\right) = \mathbf{131,3\ dB}
\end{equation}

\begin{equation}
\mathbf{131,3 \leq FSPL^{dB} \leq 144,8\ dB}
\end{equation}

\subsection{Downlink/Uplink}

Considering the frequency of the downlink/uplink as 436,9 MHz, the minimum and maximum FSBL is:

\begin{equation}
FSPL_{max} = 32,45 + 20\log\left(\frac{2830}{1\ km}\right) + 20\log\left(\frac{436,9}{1\ MHz}\right) = \mathbf{154,3\ dB}
\end{equation}

\begin{equation}
FSPL_{min} = 32,45 + 20\log\left(\frac{600}{1\ km}\right) + 20\log\left(\frac{436,9}{1\ MHz}\right) = \mathbf{140,8\ dB}
\end{equation}

\begin{equation}
    \mathbf{140,8 \leq FSPL^{dB} \leq 154,3\ dB}
\end{equation}

\subsection{Uplink (Payload)}

Considering the frequency of the payload's uplink is 401,635 MHz, the minimum and maximum FSBL is:

\begin{equation}
FSPL_{max} = 32,45 + 20\log\left(\frac{2830}{1\ km}\right) + 20\log\left(\frac{401,635}{1\ MHz}\right) = \mathbf{153,6\ dB}
\end{equation}

\begin{equation}
FSPL_{min} = 32,45 + 20\log\left(\frac{600}{1\ km}\right) + 20\log\left(\frac{401,635}{1\ MHz}\right) = \mathbf{140,1\ dB}
\end{equation}

\begin{equation}
    \mathbf{140,1 \leq FSPL^{dB} \leq 153,6\ dB}
\end{equation}

\section{Signal-to-Noise-Ratio}

The Signal-to-Noise-Ratio (SNR\nomenclature{\textbf{SNR}}{\textit{Signal To Noise Ratio}}) of a transmitted signal at the receiver can be expressed using \autoref{eq:snr}:

\begin{equation} \label{eq:snr}
SNR = \frac{E_{b}}{N_{0}} = \frac{P_{t}G_{t}G_{r}}{kT_{s}RL_{p}}
\end{equation}

Where:

\begin{itemize}
    \item $P_{t}$ = Transmitter power
    \item $G_{t}$ = Transmitter gain
    \item $G_{r}$ = Receiver gain
    \item $k$ = Boltzmann's constant ($\cong 1,3806 \times 10^{-23}\ J/K$)
    \item $T_{s}$ = System noise temperature
    \item $R$ = Data rate in bits per seconds (bps)
    \item $L_{p}$ = Free-Space Path Loss (FSPL)
\end{itemize}

The system noise temperature ($T_{s}$) can be defined using \autoref{eq:system-noise-temperature}.

\begin{equation} \label{eq:system-noise-temperature}
    T_{s} = T_{ant} + T_{r}
\end{equation}

with:

\begin{equation}
    T_{r} = \frac{T_{0}}{L_{r}} (F - L_{r})
\end{equation}

and:

\begin{equation}
    F = 1 + \frac{T_{r}}{T_{0}} = 1 + \frac{323}{290} = 2,114
\end{equation}

The SNR value in decibels can be calculated using the \autoref{eq:snr-db}:

\begin{equation} \label{eq:snr-db}
    \begin{split}
        SNR^{dB} & = 10\log_{10}\left( \frac{E_{b}}{N_{0}} \right) = 10\log_{10} \left( \frac{P_{t}G_{t}G_{r}}{kT_{s}RL_{p}} \right) \\
                 & = P_{t}^{dBm} - 30 + G_{t}^{dBi} + G_{r}^{dBi} - L_{p}^{dB} - 10\log k - 10\log T_{s} - 10\log R
    \end{split}
\end{equation}

Considering other losses in the system ($L_{s}$) (cable and connection losses as example), the \autoref{eq:snr-db} can be corrected as presented in \autoref{eq:snr-db-with-losses}.

\begin{equation} \label{eq:snr-db-with-losses}
    SNR^{dB} = P_{t}^{dBm} - 30 + G_{t}^{dBi} + G_{r}^{dBi} - L_{p}^{dB} - L_{s}^{dB} - 10\log k - 10\log T_{s} - 10\log R
\end{equation}

\subsection{Beacon}

Using Equations \ref{eq:snr-db-with-losses} and \ref{eq:system-noise-temperature}, with:

\begin{itemize}
    \item $P_{t} = 30\ dBm$
    \item $G_{t} = 0\ dBi$
    \item $G_{r} = 12\ dBi$
    \item $L_{p} = 144,8\ dB$
    \item $L_{s} = 5\ dB$
    \item $R = 1200\ bps$
    \item $T_{0} = 290\ K$
    \item $T_{r} = 323\ K$
    \item $T_{ant} = \ K$
\end{itemize}

\begin{equation}
    F = 1 + \frac{T_{r}}{T_{0}} = 1 + \frac{323}{290} = 2,114
\end{equation}

\begin{equation}
    T_{r} = \frac{T_{0}}{L_{r}} (F - L_{r}) = \frac{290}{L_{r}} (2,114 - L_{r}) = \ K
\end{equation}

\begin{equation}
    T_{s} = T_{ant} + T_{r} =  = \ K
\end{equation}

\begin{equation}
    SNR^{dB} = 30 - 30 + 0 + 12 - 144,8 - 5 + 228,6 - 30,64 - 30,79 = 30,37\ dB
\end{equation}

\begin{equation}
\mathbf{SNR^{dB} \geq 30,87\ dB}
\end{equation}

\subsection{Downlink}

Using Equations \ref{eq:snr-db-with-losses} and \ref{eq:system-noise-temperature}, with:

\begin{itemize}
    \item $P_{t} = 30\ dBm$
    \item $G_{t} = 0\ dBi$
    \item $G_{r} = 15\ dBi$
    \item $L_{p} = 154,3\ dB$
    \item $L_{s} = 5\ dB$
    \item $R = 4800\ bps$
    \item $T_{0} = 290\ K$
    \item $T_{r} = \ K$
    \item $T_{ant} = \ K$
\end{itemize}

\begin{equation}
    SNR^{dB} = 30 - 30 + 0 + 15.5 - 154,3 - 5 + 228,6 - 30,64 - 36,81 = 17,35\ dB
\end{equation}

\begin{equation}
\mathbf{SNR^{dB} \geq 17,35\ dB}
\end{equation}

\subsection{Uplink}

Using Equations \ref{eq:snr-db-with-losses} and \ref{eq:system-noise-temperature}, with:

\begin{itemize}
    \item $P_{t} = 30\ dBm$
    \item $G_{t} = 15\ dBi$
    \item $G_{r} = 0\ dBi$
    \item $L_{p} = 154,3\ dB$
    \item $L_{s} = 7\ dB$
    \item $R = 4800\ bps$
    \item $T_{0} = 290\ K$
    \item $T_{r} = \ K$
    \item $T_{ant} = \ K$
\end{itemize}

\begin{equation}
    SNR^{dB} = 47 - 30 + 15.5 + 0 - 154,3 - 7 + 228,6 - 31,39 - 36,81 = 31,60\ dB
\end{equation}

\begin{equation}
    \mathbf{SNR^{dB} \geq 31,60\ dB}
\end{equation}

\subsection{Uplink (Payload)}

TBD

\section{Link Margin}

From \cite{larson2005}, the minimum SNR value at the received considering a $10^{-5}$ bit error rate is:

\begin{itemize}
    \item Beacon: $SNR^{dB} \geq 9,6\ dB$
    \item Downlink/Uplink: $SNR^{dB} \geq 9,6\ dB$
    \item Uplink (payload): $SNR^{dB} \geq XX\ dB$
\end{itemize}

And considering the link margin as the SNR of the link minus the SNR threshold for a given bit error, the link margin of the radio links of the satellite are:

\begin{itemize}
    \item Beacon: $30,87 - 9,6 = \mathbf{21,27\ dB}$
    \item Downlink: $17,35 - 9,6 = \mathbf{7,75\ dB}$
    \item Uplink: $31,60 - 9,6 = \mathbf{22\ dB}$
    \item Uplink (payload): $X - X = \mathbf{XX\ dB}$
\end{itemize}
